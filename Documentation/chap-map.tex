\chapter{Map}

In the context of \sysname{}, a \emph{map} is a class that maps
each \emph{key} in a set of keys to some kind of object.  The object
being mapped to can either be another map, or some \emph{leaf
  object}.  Since maps can be nested this way, one can think of
\sysname{} as mapping \emph{sequences} of keys to such \emph{leaf
  objects}.

\Defprotoclass {map}

This class is the protocol class for all maps.

Client code can use \texttt{make-instance}, passing this class or the
name of this class as an argument.

\Definitarg {:test}

Client code can pass this initialization argument to
\texttt{make-instance} in order to choose a function for comparing
keys in the map.  The default value of this initialization argument is
the function named \texttt{eql}.

\sysname{} automatically chooses an implementation class based on the
value of the \texttt{:test} initialization argument.  If the value is
a function or the name of a function that is acceptable to
\texttt{make-hash-table}, then \sysname{} returns a map that uses a
hash table.  Otherwise it returns a map using an association list.

\Defclass {hash-table-map}

This class is a subclass of the class \texttt{map}.  It is returned
from a call to \texttt{make-instance} when either no initialization
argument is supplied, or when the initialization argument
\texttt{:test} is supplied, and its value is acceptable as the
\texttt{:test} argument for \texttt{make-hash-table}.

\Defclass {alist-map}

This class is a subclass of the class \texttt{map}.  It is returned
from a call to \texttt{make-instance} when the initialization argument
\texttt{:test} is supplied, and its value is \emph{not} acceptable as
the \texttt{:test} argument for \texttt{make-hash-table}.
