\chapter{Searching maps}

Client code does not manipulate maps directly.  Instead, it uses some
\emph{state} object supplied by \sysname{}.

\Defprotoclass {state}

This class is the protocol class for all states.

\Defgeneric {path} {state}

This generic function returns a \emph{path}, i.e. a list of keys that
have been supplied so far, and that resulted in \texttt{state}.

\Defclass {internal-state}

This class is a subclass of the class \texttt{state}.  It represents a
\emph{partial result} in that the sequence of keys supplied to the
search so far corresponds to a map, and not to a leaf object.

\Defgeneric {map} {internal-state}

Given an internal state, this generic function returns the map
associated with that state.

\Defclass {final-state}

This class is a subclass of the class \texttt{state}.  It represents a
\emph{final result} in that the sequence of keys supplied so far
corresponds to a leaf object.

\Defgeneric {object} {final-state}

Given a final state, this generic function returns the leaf object
associated with that state.

\Defclass {error-state}

This class is a subclass of the class \texttt{state}.  It represents a
situation where the sequence of keys supplied so far contains at least
one key that is associated with neither another map nor with a leaf
object in the corresponding map.

\Defun {make-initial-state} {map}

This function creates an instance of the class
\texttt{internal-state}.  The map associated with that internal state
(as returned by the generic function \texttt{map}) is the argument
\texttt{map}, and the path associated with that internal state (as
returned by the generic function \texttt{path}) is the empty list.
